\section{Contexto}

Se desea crear un programa capaz de elegir una cantidad predeterminada de puntos, entre un conjunto dado de los cuales, de forma que se ocupe el mayor espacio posible (es decir que las distancias entre los puntos sean tan grandes como se pueda). Se consigue esto a traves de varios algoritmos, los cuales imprimen sus resultados posteriormente por consola. 

Cogeremos los resultados obtenidos por cada algoritmo (como los obtenidos por cada cambio pequeño dentro de un mismo algoritmo) para posteriormente evaluar estos y compararlos entre si (de modo que podamos determinar cuales son los más rápidos, los que proveen mejores resultados y los que son más eficientes en ambos campos).

\subsection{Objetivos}

Durante el desarrollo de esta práctica, se han cumplido los objetivos listados a continuación:
\begin{itemize}
   \item Estructura para leer los contenidos de un directorio y repartir estos a varios algoritmos
   \item Método facil de agregar tests adicionales
   \item Diseño e implementación de un algoritmo voraz
   \item Diseño e implementación de un algoritmo de búsqueda local
   \item Diseño e implementación de un algoritmo GRASP para el problema
   \item Diseño e implementación de un algoritmo de búsqueda tabú con memoria a corto plazo
   \item Diseño e implementación de un algoritmo de ramificación y poda
   \item Ampliar el apartado anterior para usar una búsqueda en profundidad como estrategia de ramificación
\end{itemize}

\section{Motivación}

Ampliar conocimientos sobre algoritmos de optimización, asi como mejorar las competencias informaticas. Además, conviene trabajar en un proyecto de una mayor escala a lo que uno esta acostumbrado dentor de la universidad ya que reflejará más realisticamente con lo que uno se encontrara en la vida real.