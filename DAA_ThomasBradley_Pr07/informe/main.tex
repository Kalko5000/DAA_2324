\documentclass[spanish,a4paper,12pt,oneside]{extreport}

\usepackage[dvips]{graphicx}
\usepackage[dvips]{epsfig}
\usepackage[utf8]{inputenc}
\usepackage[spanish]{babel}
\usepackage{alltt}

\usepackage[ruled,vlined,commentsnumbered,linesnumbered,inoutnumbered,titlenotnumbered,noend]{algorithm2e}
\SetKwRepeat{Do}{do}{while}

\usepackage{multirow}
\usepackage{array} 
\usepackage{amsfonts}
\usepackage{amsmath}
\usepackage{bigstrut}
\usepackage{booktabs}
\usepackage{caption}
\usepackage{chngpage}
\usepackage{float}
\usepackage{enumitem,lipsum}
\usepackage{graphicx}
\usepackage{lscape}
\usepackage{microtype}
\usepackage{natbib}
\usepackage{pdflscape}
\usepackage{rotating}
\usepackage{subcaption}
\usepackage{ctable}
\usepackage[hidelinks]{hyperref}
%\usepackage{enumerate}
\usepackage{gensymb}
\usepackage{eurosym}
\usepackage{xcolor}
\usepackage{tabu}

\usepackage{lineno}
%\\linenumbers
%\setlength\linenumbersep{5pt}
%\renewcommand\linenumberfont{\normalfont\tiny\sffamily\color{gray}}

\usepackage[top=2cm, bottom=2cm, left=2cm, right=2cm]{geometry}

\newenvironment{sourcecode}
{\begin{list}{}{\setlength{\leftmargin}{1em}}\item\scriptsize\bfseries}
{\end{list}}

\newenvironment{littlesourcecode}
{\begin{list}{}{\setlength{\leftmargin}{1em}}\item\tiny\bfseries}
{\end{list}}

\newenvironment{summary}
{\par\noindent\begin{center}\textbf{Abstract}\end{center}\begin{itshape}\par\noindent}
{\end{itshape}}

\newenvironment{keywords}
{\begin{list}{}{\setlength{\leftmargin}{1em}}\item[\hskip\labelsep \bfseries Keywords:]}
{\end{list}}

\newenvironment{palabrasClave}
{\begin{list}{}{\setlength{\leftmargin}{1em}}\item[\hskip\labelsep \bfseries Palabras clave:]}
{\end{list}}

\usepackage{bera}% optional: just to have a nice mono-spaced font
\usepackage{listings}
\usepackage{xcolor}

\colorlet{punct}{red!60!black}
\definecolor{background}{HTML}{EEEEEE}
\definecolor{delim}{RGB}{20,105,176}
\colorlet{numb}{magenta!60!black}

\lstdefinelanguage{json}{
    basicstyle=\normalfont\ttfamily,
    numbers=left,
    numberstyle=\scriptsize,
    stepnumber=1,
    numbersep=8pt,
    showstringspaces=false,
    breaklines=true,
    frame=lines,
    backgroundcolor=\color{background},
    literate=
     *{0}{{{\color{numb}0}}}{1}
      {1}{{{\color{numb}1}}}{1}
      {2}{{{\color{numb}2}}}{1}
      {3}{{{\color{numb}3}}}{1}
      {4}{{{\color{numb}4}}}{1}
      {5}{{{\color{numb}5}}}{1}
      {6}{{{\color{numb}6}}}{1}
      {7}{{{\color{numb}7}}}{1}
      {8}{{{\color{numb}8}}}{1}
      {9}{{{\color{numb}9}}}{1}
      {:}{{{\color{punct}{:}}}}{1}
      {,}{{{\color{punct}{,}}}}{1}
      {\{}{{{\color{delim}{\{}}}}{1}
      {\}}{{{\color{delim}{\}}}}}{1}
      {[}{{{\color{delim}{[}}}}{1}
      {]}{{{\color{delim}{]}}}}{1},
}

\begin{document}

\renewcommand\listtablename{Índice de Tablas}    
\renewcommand\listfigurename{Índice de Figuras}    

%%%%%%%%%%%%%%%%%%%%%%%%%%%%%%%%%%%%%%%%%%%%%%%%%%%%%%%%%%%%%%%%%%%%%%%%%%%%%%%
% First Page
%%%%%%%%%%%%%%%%%%%%%%%%%%%%%%%%%%%%%%%%%%%%%%%%%%%%%%%%%%%%%%%%%%%%%%%%%%%%%%%
\pagestyle{empty}
\thispagestyle{empty}


\newcommand{\HRule}{\rule{\linewidth}{1mm}}
\setlength{\parindent}{0mm}
\setlength{\parskip}{0mm}

\vspace*{\stretch{0.5}}

\begin{center}
\includegraphics[scale=0.8]{images/escuela-ingenieria-tecnologia-original}\\[10mm]
{\Huge Informe de la Práctica 7 de DAA \\[3mm] Curso 2023-2024}
\end{center}

\HRule
\begin{flushright}
        {\Huge \itshape Parallel Machine Scheduling Problem \\ with Dependent Setup Times} \\[5mm]
        {\Large Thomas Edward Bradley} \\[5mm]


\end{flushright}
\HRule
\vspace*{\stretch{2}}
\begin{center}
  \Large La Laguna, \today
\end{center}

\setlength{\parindent}{5mm}

\newpage
La Figura \ref{fig:PMSP} muestra una imagen creada por Midjourney recreando un ejemplo de aplicación real del problema abordado en esta práctica.

\begin{figure}
    \centering
    \includegraphics[scale=0.4]{images/PMSP_Midjourney.png}
    \caption{Midjourney}
    \label{fig:PMSP}
\end{figure}

%%%%%%%%%%%%%%%%%%%%%%%%%%%%%%%%%%%%%%%%%%%%%%%%%%%%%%%%%%
\newpage
\thispagestyle{empty}

{ \flushright

\begin{LARGE}
Agradecimientos
\end{LARGE}

\hspace{3mm}

\begin{large}
Maria Belen Melian Batista

\bigskip

\end{large}

}
%%%%%%%%%%%%%%%%%%%%%%%%%%%%%%%%%%%%%%%%%%%%%%%%%%%%%%%%
\newpage
\thispagestyle{empty}

\bigskip
\begin{LARGE}
Licencia
\end{LARGE}

\begin{center}
\includegraphics[scale=1.8]{images/by-nc_88x31}\\[5mm]
\end{center}

\begin{large}
© Esta obra está bajo una licencia de Creative Commons Reconocimiento-NoComercial 4.0 Internacional.
\end{large}

%%%%%%%%%%%%%%%%%%%%%%%%%%%%%%%%%%%%%%%%%%%%%%%%%%%%%%%%
\newpage
\thispagestyle{empty}

%%%%%%%%%%%%%%%%%%%%%%%%%%%%%%%%%%%%%%%%%%%%%%%%%%%%%%%%
\newpage 
\thispagestyle{empty}

\begin{abstract}
{\em

El siguiente informe viene a redactar los conceptos, pseudocoódigos y resultados obtenidos al evaluar distribuciones optimas de tareas entre varias maquinas. El número tanto de tareas como máquinas es variable y cada tarea tiene asociado un tiempo de ejecución y varios tiempos de transición a dicha tarea partiendo de cualquier otro.
}

\bigskip

\begin{palabrasClave}
Scheduling optimization problem, Total Completion Time, Machine, Task, Greedy, GRASP, LS, VNS, GVNS. 
\end{palabrasClave}


\end{summary}
%%%%%%%%%%%%%%%%%%%%%%%%%%%%%%%%%%%%%%%%%%%%%%%%%%%%%%%%%
\newpage{\pagestyle{empty}}
\thispagestyle{empty}

%%%%%%%%%%%%%%%%%%%%%%%%%%%%%%%%%%%%%%%%%%%%%%%%%%%%%%%%%
\pagestyle{myheadings} %my head defined by markboth or markright
% No funciona bien \markboth sin "twoside" en \documentclass, pero al
% ponerlo se dan un montón de errores de underfull \vbox, con lo que no se
% ha puesto.


%%Aqui debería poner el nombre del proyecto pero, como es muy grande no cabe y se ve feo en el PDF
\markboth{xxxxx}{}

%%%%%%%%%%%%%%%%%%%%%%%%%%%%%%%%%%%%%%%%%%%%%%%%%%%%%%%%%
%Numeracion en romanos
\renewcommand{\thepage}{\roman{page}}
\setcounter{page}{1}
\pagestyle{plain} 

%%%%%%%%%%%%%%%%%%%%%%%%%%%%%%%%%%%%%%%%%%%%%%%%%%%%%%%%%

\tableofcontents

%%%%%%%%%%%%%%%%%%%%%%%%%%%%%%%%%%%%%%%%%%%%%%%%%%%%%%%%%
\newpage{\pagestyle{empty}}

\listoffigures

%%%%%%%%%%%%%%%%%%%%%%%%%%%%%%%%%%%%%%%%%%%%%%%%%%%%%%%%%
\newpage{\pagestyle{empty}}

\listoftables

%%%%%%%%%%%%%%%%%%%%%%%%%%%%%%%%%%%%%%%%%%%%%%%%%%%%%%%%%%%%%%%%%%%%%%%%%%%%%%%
\newpage{\pagestyle{empty}}

%%%%%%%%%%%%%%%%%%%%%%%%%%%%%%%%%%%%%%%%%%%%%%%%%%%%%%%%%%%%%%%%%%%%%%%%%%%%%%%
\newpage
\thispagestyle{empty}

%Numeracion a partir del capitulo I
\renewcommand{\thepage}{\arabic{page}}
\setcounter{page}{1}
\pagestyle{plain}

\chapter{\LARGE Introducción}
\label{chapter:intro}

\section{Contexto}

Se deseo crear un programa que pueda resolver un problema PMS a traves de varios algoritmos. Devolviendo una tabla para cada uno de estos para permitir una posterior evaluación y comparación de los resultados obtenidos.

\subsection{Objetivos}

Durante el desarrollo de esta práctica, se han cumplido los objetivos listados a continuación:
\begin{itemize}
   \item Almacenamiento de un problema PMS y posterior calculo de la matriz t
   \item Desarrollo de un algoritmo Voraz para la resolución del problema
   \item Muestreo de resultados mediante tablas ordenadas, conteniendo la ejecuión para un directorio dado
   \item Facil implementación de cualquier otra tabla que se desee evaluar
   \item Desarrollo de un algoritmo GRASP para la resolución del problema
   \item Desarrollo de un algoritmo GVNS para la resolución del problema
   \item Las 4 busquedas locales empleadas en GRASP y GVNS
   \item Una pequeña optimización en las busquedas locales, empleando una evaluación del TCT a nivel de maquina en lugar de a nivel global (no se implemento la manera más optima en este caso)
\end{itemize}

\section{Motivación}

Trabajar en un proyecto de mayor escala para mejorar las competencias informaticas en un entorno más cercano a la programación en la vida real.

%%%%%%%%%%%%%%%%%%%%%%%%%%%%%%%%%%%%%%%%%%%%%%%%%%%%%%%%%%%%%%%%%%%%%%%%%%%%%%%
\newpage{\pagestyle{empty}}
\thispagestyle{empty}

\chapter{\LARGE Parallel Machine Scheduling Problem \\ with Dependent Setup Times}
\label{chapter:Problema}

\section{Descripción}\label{Problem_PMP}

En esta pr\'actica estudiaremos un problema de secuenciaci\'on de tareas en m\'aquinas paralelas con tiempos de setup dependientes; Parallel Machine Scheduling Problem
with Dependent Setup Times \cite{Baez2019}. El objetivo del problema es asignar tareas a las m\'aquinas y determinar el orden en el que deben ser procesadas en las m\'aquinas de tal manera que la suma de los tiempos de finalizaci\'on de todos los trabajos, es decir, el tiempo total de finalizaci\'on (\textit{TCT}), sea minimizado.

El tiempo de setup es el tiempo necesario para preparar los recursos necesarios (personas, m\'aquinas) para realizar una tarea (operaci\'on, trabajo). En algunas situaciones, los tiempos de setup var\'ian seg\'un la secuencia de trabajos realizados en una m\'aquina; por ejemplo en las industrias qu\'imica, farmac\'eutica y de procesamiento de metales, donde se deben realizar tareas de limpieza o reparaci\'on para preparar el equipo para realizar la siguiente tarea.

Existen varios criterios de desempe\~no para medir la calidad de una secuenciaci\'on de tareas dada. Los criterios m\'as utilizados son la minimizaci\'on del tiempo m\'aximo de finalizaci\'on (\textit{makespan}) y la minimizaci\'on del \textit{TCT}. En particular, la minimizaci\'on del \textit{TCT} es un criterio que contribuye a la maximizaci\'on del flujo de producci\'on, la minimizaci\'on de los inventarios en proceso y el uso equilibrado de los recursos.

El problema abordado en esta pr\'actica tiene las siguientes caracter\'isticas:
\begin{itemize}
	\item Se dispone de \textit{m} m\'aquinas paralelas id\'enticas que est\'an continuamente disponibles.
  \item Hay \textit{n} tareas independientes que se programar\'an en las m\'aquinas. Todas las tareas est\'an disponibles en el momento cero.
  \item Cada m\'aquina puede procesar una tarea a la vez sin preferencia y deben usarse todas las m\'aquinas.
	\item Cualquier m\'aquina puede procesar cualquiera de las tareas.
	\item Cada tarea tiene un tiempo de procesamiento asociado $p_j$.
	\item Hay tiempos de setup de la m\'aquina $s_{ij}$ para procesar la tarea $j$ justo despu\'es de la tarea $i$, con $s_{ij} \neq s_{ji}$, en general. Hay un tiempo de setup $s_{0j}$ para procesar la primera tarea en cada m\'aquina.
  \item El objetivo es minimizar la suma de los tiempos de finalizaci\'on de los trabajos, es decir, minimizar el TCT.
\end{itemize}

El problema consiste en asignar las \textit{n} tareas a las \textit{m} m\'aquinas y determinar el orden en el que deben ser procesadas de tal manera que se minimice el TCT.

El problema se puede definir en un grafo completo $G = (V,A)$, donde \linebreak $V = \{0,1,2, \cdots, n\}$ es el conjunto de nodos y $A$ es el conjunto de arcos. El nodo $0$ representa el estado inicial de las m\'aquinas (trabajo ficticio) y los nodos del conjunto $I = \{1,2, \cdots, n\}$ corresponden a las tareas. Para cada par de nodos $i, j \in V$, hay dos arcos $(i, j), (j, i) \in A$ que tienen asociados los tiempos de setup $s_{ij}, s_{ji}$ seg\'un la direcci\'on del arco. Cada nodo $j \in V$ tiene asociado un tiempo de procesamiento $p_j$ con $p_0 = 0 $. Usando los tiempos de setup $s_{ij}$ y los tiempos de procesamiento $p_j$, asociamos a cada arco $(i, j) \in A$ un valor $t_{ij} = s_{ij} + p_j, (i \in V, j \in I)$.

Sea $P_r = \{0, [1_r], [2_r], \cdots, [k_r] \}$ una secuencia de $k_r + 1$ tareas en la m\'aquina \textit{r} con el trabajo ficticio $0$ en el posici\'on cero de $P_r$, donde $[i_r]$ significa el nodo (tarea) en la posici\'on $i_r$ en la secuencia \textit{r}. Luego, el tiempo de finalizaci\'on $C_{[i_r]}$ del trabajo en la posici\'on $i_r$ se calcula como $C_{[i_r]} = \sum_ {j = 1} ^ {i_r} t _ {[j-1 ] [j]} $. Tenga en cuenta que en el grafo \textit{G} representa la longitud de la ruta desde el nodo $0$ al nodo $[i_r]$.

Sumando los tiempos de finalizaci\'on de los trabajos en $P_r$ obtenemos la suma de las longitudes de las rutas desde el nodo 0 a cada nodo en $P_r$ ($TCT(P_r)$) como:

\begin{align}
\label{form1}TCT(P_r) = \sum_{i=1}^k C_{[i]} =  
kt_{[0][1]} + (k-1) t_{[1][2]} + \nonumber \\ \cdots + 2t_{[k-2][k-1]} + t_{[k-1][k]}
\end{align}

Usando lo anterior, el problema se puede formular como encontrar \textit{m} rutas simples disjuntas en $G$ que comienzan en el nodo ra\'iz 0, que juntas cubren todos los nodos en \textit{I} y minimizan la funci\'on objetivo.

\begin{align}
\label{form2}z= \sum_{r=1}^{m} TCT(P_r ) =  \sum_{r=1}^{m}\sum_{i=1_{r}}^{k_{r}}(k_{r}-i+1)t_{[i-1][i]}
\end{align}

Tenga en cuenta que el coeficiente $(k_{r} - i + 1)$ indica el n\'umero de nodos despu\'es del nodo en la posici\'on $i_r-1 $.

%Por ejemplo, una soluci\'on para el problema con 10 tareas con tiempos de procesamiento {t1(2),t2(3), t3(5), t4(8), t5(1), t6(4), t7(6), t8(9), t9(10), t10(4)} y 2 m\'aquinas, podr\'ia ser la siguiente: M\'aquina 1 - tareas 9, 10, 8 y 6; M\'aquina 2 - tareas 3, 5, 4, 7, 1 y 2; con un TCT de 92 (TCT m\'aquina 1 = (10 + (10+4) + (10+4+9) + (10+4+9+4)) = 74; TCT m\'aquina 2 = (5 + (5+1) + (5+1+8) + (5+1+8+6) + (5+1+8+6+2) + (5+1+8+6+2+3)) = 92). 


\subsection{Representaci\'on de las soluciones}

Podemos generar un array por cada una de las m\'aquinas, S=\{$A_1$, $A_2$,.., $A_m$\}. En ellos se insertar\'an las tareas a ser procesadas en cada m\'aquina en el orden establecido. 


%%%%%%%%%%%%%%%%%%%%%%%%%%%%%%%%%%%%%%%%%%%%%%%%%%%%%%%%%%%%%%%%%%%%%%%%%%%%%%%%%%%%%%%%%%%%%%%%%%%

%%%%%%%%%%%%%%%%%%%%%%%%%%%%%%%%%%%%%%%%%%%%%%%%%%%%%%%%%%%%%%%%%%%%%%%%%%%%%%%
\newpage{\pagestyle{empty}}
\thispagestyle{empty}

\chapter{\LARGE Algoritmos}
\label{chapter:algoritmos}

    \section{Algoritmo consturictivo voraz}

%%%%%%%%%%%%%%%%%%%%%%%%%%%%%%%%%%%%%%%%%%%%%%%%%%%%%%%%%%%%%%%%%%%%%%%%%%%%%%%%%%%%%%%%%%%%%%%%%%%
\section*{Un constructivo voraz}
Un algoritmo constructivo voraz muy sencillo para este problema parte del subconjunto, $S$, formado por las $m$ tareas con menores valores de $t_{0j}$ asignadas a los $m$ arrays que representan la secuenciaci\'on de tareas en las m\'aquinas. A continuaci\'on, a\~nade a este subconjunto, iterativamente, la tarea-m\'aquina-posici\'on que menor incremento produce en la funci\'on objetivo. El pseudoc\'odigo de este algoritmo se muestra a conti\-nuaci\'on.
%%%%%%%%%%%%%%%%%%%%%%%%%%%%%%%%%%%%%%
\begin{figure}[h!]
{\small
 \hrule \
 {\bf\small Algoritmo constructivo voraz}
 \hrule
\begin{center}
\begin{tabbing}
\ 1: Seleccionar la $m$ tareas $j_1, j_2,.., j_m$ con menores valores de $t_{0j}$ para ser introducidas en las \\ primeras posiciones de los arrays que forman la solucion $S$;\\
\ 2: S = \{A_1=\{j_1\}, A_2=\{j_2\},.., A_m=\{j_m\}\};\\
\ 3: {\bf rep}\={\bf eat}\\
\ 4: \> $S^* = S$;\\
\ 5: \> Obtener la tarea-maquina-posicion que minimiza el incremento del $TCT$;\\
\ 6: \> Insertarla en la posicion que corresponda y actualizar $S^*$;\\
\ 7: {\bf until} (todas las tareas han sido asignadas a alguna maquina)\\
\ 9: Devolver $S^*$;
\end{tabbing}
\end{center}
\hrule
}
\caption{Algoritmo constructivo voraz}
\label{constructivo}
\end{figure}

%%%%%%%%%%%%%%%%%%%%%%%%%%%%%%%%%%%%%%%%%%%%%%%%%%%%%%%%%%%%
%%%%%%%%%%%%%%%%%%%%%%%%%%%%%%%%%%%%%%%%%%%%%%%%%%%%%%%%%%%%
\section{Búsquedas Locales}

Se han implementado 4 busquedas locales, las cuales se detallaran a continuación:
\begin{itemize}
  \item \textbf{Inserción Interna} - Trata de insertar cada tarea de una maquinas en todas las posiciones restantes de la misma
  \item \textbf{Inserción Externa} -Trata de insertar cada tarea de las maquinas en todas las posiciones posibles en otras maquinas
  \item \textbf{Intercambio Interno} - Trata de intercambiar todos las tareas de una maquian entre si
  \item \textbf{Intercambio Externo} - Trata de intercambiar todas las tareas de las maquinas por todas las tareas de otras
\end{itemize}
 
%%%%%%%%%%%%%%%%%%%%%%%%%%%%%%%%%%%%%%%%%%%%%%%%%%%%%%%%%%%%
%%%%%%%%%%%%%%%%%%%%%%%%%%%%%%%%%%%%%%%%%%%%%%%%%%%%%%%%%%%%
\section{GRASP}
Para el algoritmo GRASP empleado, tras experimentar con varios tanmaños de instancias (veces que corremos la fase constructiva) opte por 10 ya que era un buen balance entre buenos resultados y una ejecución rápida.

\begin{figure}[h!]
{\small
 \hrule \
 {\bf\small Algoritmo constructivo voraz}
 \hrule
\begin{center}
\begin{tabbing}
\textbf{Procedure} \= \textbf{GRASP} \\
\textbf{Begin} \\
\> Preprocesamiento \\
\> \textbf{Repeat} \\
\> \> Fase Constructiva(Solución); Usando para ello una lista de n candidatos \\
\> \> PostProcesamiento(Solución); Miramos optimos locales para obtener una nueva solución \\
\> \> Actualizar(Solución, MejorSolución); Si es mejor que la mejor encontrada, se guarda \\
\> \textbf{Until} (Se halla llegado a las iteraciones deseadas); \\
\textbf{End.} \\
\end{tabbing}
\end{center}
\hrule
}
\caption{Algoritmo constructivo grasp}
\label{constructivo}
\end{figure}


%%%%%%%%%%%%%%%%%%%%%%%%%%%%%%%%%%%%%%%%%%%%%%%%%%%%%%%%%%%%
%%%%%%%%%%%%%%%%%%%%%%%%%%%%%%%%%%%%%%%%%%%%%%%%%%%%%%%%%%%%
\section{GVNS}

En cuanto a las instancias (veces que corremos la fase constructiva) opte por 100, ya que era un buen número para obtener resultados y no tardaba una cantidad de tiempo excesivo (por no optimizar las busquedas locales tanto como podria, llevar  acabo las 1000 iteraciones simplemente llevaba demasiado tiempo).

%%%%%%%%%%%%%%%%%%%%%%%%%%%%%%%%%%%%%%%%%%%%%%%%%%%%%%%%%
\newpage{\pagestyle{empty}}
\thispagestyle{empty}

\chapter{\LARGE Experimentos y resultados computacionales}
\label{chapter:resultados}

%%%%%%%%%%%%%%%%%%%%%%%%%%%%%%%%%%%%%%%%%%%%%%%%%%%%%%%%%%%%%%
%%%%%%%%%%%%%%%%%%%%%%%%%%%%%%%%%%%%%%%%%%%%%%%%%%%%%%%%%%%%%%
\section{Constructivo voraz}



%%%%%%%%%%%%%%%%%%%%%%%%%%%%
   \begin{table}[h]
   {\small
   \begin{center}
   \begin{tabular}{cccccc}
      \multicolumn{6}{c}{Algoritmo voraz} \\
      \hline
      Problema & $n$ & $m$ &  Ejecuci\'on & $TCT$ & $CPU$ \\
      \hline
      $ID$   &     &    &     $1$      &       &       \\
      $ID$   &     &    &     $2$      &       &       \\
      $ID$   &     &    &     $3$      &       &       \\
      $ID$   &     &    &     $4$      &       &       \\
      $ID$   &     &    &     $5$      &       &       \\        
      $\cdots$ &$\cdots$ & $\cdots$ &$\cdots$ &$\cdots$ &$\cdots$ \\
      \hline
   \end{tabular}
   \end{center}
   }
   \caption{Algoritmo voraz. Tabla de resultados}
   \end{table}


%%%%%%%%%%%%%%%%%%%%%%%%%%%%%%%%%%%%%%%%%%%%%%%%%%%%%%%%%%%%%%
%%%%%%%%%%%%%%%%%%%%%%%%%%%%%%%%%%%%%%%%%%%%%%%%%%%%%%%%%%%%%%
\section{Multiarranque}

Comparativa probando diferentes búsquedas locales.
Debe incluir el diseño de experimentos para el ajsute de parámetros y el análisis de los resultados obtenidos

   \begin{table}[h]
   {\small
   \begin{center}
   \begin{tabular}{ccccc}
      \multicolumn{5}{c}{Multiarranque} \\
      \hline
      Problema & $n$ &  Ejecuci\'on & $TCT$ & $CPU$ \\
      \hline
      $ID$   &     &     $1$      &       &       \\
      $ID$   &     &     $2$      &       &       \\
      $ID$   &     &     $3$      &       &       \\
      $ID$   &     &     $4$      &       &       \\
      $ID$   &     &     $5$      &       &       \\      
      $\cdots$ &$\cdots$ &$\cdots$ &$\cdots$ &$\cdots$ \\
      \hline
   \end{tabular}
   \end{center}
   }
   \caption{Multiarranque. Tabla de resultados}
   \end{table}


%%%%%%%%%%%%%%%%%%%%%%%%%%%%%%%%%%%%%%%%%%%%%%%%%%%%%%%%%%%%%%
%%%%%%%%%%%%%%%%%%%%%%%%%%%%%%%%%%%%%%%%%%%%%%%%%%%%%%%%%%%%%%
\section{GRASP}

Comparativa probando diferentes búsquedas locales.
Debe incluir el diseño de experimentos para el ajsute de parámetros y el análisis de los resultados obtenidos.

   \begin{table}[h]
   {\small
   \begin{center}
   \begin{tabular}{ccccccc}
      \multicolumn{6}{c}{GRASP} \\
      \hline
      Problema & $n$ &  $|LRC|$ & Ejecuci\'on & $TCT$ & $CPU$ \\
      \hline
      $ID$   &     &    $2$   & $1$      &       &       \\
      $ID$   &     &    $2$   & $2$      &       &       \\
      $ID$   &     &    $2$   & $3$      &       &       \\
      $ID$   &     &    $2$   & $4$      &       &       \\
      $ID$   &     &    $2$   & $5$      &       &       \\
      $ID$   &     &    $3$   & $1$      &       &       \\
      $ID$   &     &    $3$   & $2$      &       &       \\
      $ID$   &     &    $3$   & $3$      &       &       \\
      $ID$   &     &    $3$   & $4$      &       &       \\
      $ID$   &     &    $3$   & $5$      &       &       \\      
      $\cdots$ &$\cdots$ &$\cdots$ &$\cdots$ &$\cdots$ &$\cdots$ \\
      \hline
   \end{tabular}
   \end{center}
   }
   \caption{GRASP. Tabla de resultados}
   \end{table}



%%%%%%%%%%%%%%%%%%%%%%%%%%%%%%%%%%%%%%%%%%%%%%%%%%%%%%%%%%%%%%
%%%%%%%%%%%%%%%%%%%%%%%%%%%%%%%%%%%%%%%%%%%%%%%%%%%%%%%%%%%%%%
\section{GVNS}
Debe incluir el diseño de experimentos para el ajsute de parámetros y el análisis de los resultados obtenidos



%%%%%%%%%%%%%%%%%%%%%%%%%%%%%%%%%%%%%%%%%%%%%%%%%%%%%%%%%%%%%%
%%%%%%%%%%%%%%%%%%%%%%%%%%%%%%%%%%%%%%%%%%%%%%%%%%%%%%%%%%%%%%
\section{Análisis compartativo entre algoritmos}


   \begin{table}[h]
   {\small
   \begin{center}
   \begin{tabular}{cccccc}
      \multicolumn{6}{c}{VNS} \\
      \hline
      Problema & $m$ &  $k_{max}$ & Ejecuci\'on & $TCT$ & $CPU$ \\
      \hline
      $ID$   &     &      $2$   & $1$      &       &       \\
      $ID$   &     &      $2$   & $2$      &       &       \\
      $ID$   &     &      $2$   & $3$      &       &       \\
      $ID$   &     &      $2$   & $4$      &       &       \\
      $ID$   &     &      $2$   & $5$      &       &       \\
      $ID$   &     &      $3$   & $1$      &       &       \\
      $ID$   &     &      $3$   & $2$      &       &       \\
      $ID$   &     &      $3$   & $3$      &       &       \\
      $ID$   &     &      $3$   & $4$      &       &       \\
      $ID$   &     &      $3$   & $5$      &       &       \\      
      $\cdots$ &$\cdots$ &$\cdots$ &$\cdots$ &$\cdots$ &$\cdots$ \\
      \hline
   \end{tabular}
   \end{center}
   }
   \caption{GVNS. Tabla de resultados}
   \end{table}
\end{enumerate}







%%%%%%%%%%%%%%%%%%%%%%%%%%%%%%%%%%%%%%%%%%%%%%%%%%%%%%%%%
\newpage{\pagestyle{empty}}
\thispagestyle{empty}

\chapter{\LARGE Conclusiones y trabajo futuro}
\label{chapter:conclusiones}

Si modificamos el algoritmo de ramificación y poda para evaluar todos los posibles caminos y soluciones. Vemos que el GRASP acierta en los mismos (ya sea por búsqueda local o búsqeuda tabú). De las dos opciones disponibles para ello, la búsqueda local produce los mismos resultados en menos tiempo por lo que dado el analísis de los algoritmos desarrollados es el más eficiente, en segundo lugar siendo la búsqueda tabú que obtiene los mismos resultados pero tardando más tiempo. Es por esto que el GRASP con búsqueda local es el mejor algoritmo para la precisión del resultado.
\\
\\
En cuanto a los algoritmos de ramificación y poda, las dos estrategias de ramificación producen resultados bastante similares. Sin embargo, la estrategia de ramificación que expande
el nodo con cota superior más pequeña resulta ser el más eficaz en tiempo (analizando los casos con cota inferior inicial generada por GRASP ya que son más precisos) por lo que resultaria el más eficaz. Si analizamos los tiempos y los resultados resulta que el algoritmo de ramificación sirve como un buen punto medio entre el voraz y el GRASP (tanto en tiempo como resultados). Incluso tras realizar el informe implemente un pequeño cambio al código que pilla el centro del conjunto dentro del cálculo de la cota superior, el cual mejora la calidad del resultado final sin realmente emperorar el tiempo, lo cual solidifica su posición como el algoritmo más eficiente globalmente (teniendo en cuenta la exactitud del resultado y el tiempo a la vez)
\\
\\
Gracias a la práctica también puedo concluir que en el caso de implementar un algoritmo que usa una búsqueda por las ramas de un árbol, conviene guardar el resultado dentro de un vector binaria dado que simplifica bastante el proceso de ramificación. 

%%%%%%%%%%%%%%%%%%%%%%%%%%%%%%%%%%%%%%%%%%%%%%%%%%%%%%%%%
\newpage{\pagestyle{empty}}
\thispagestyle{empty}

%%%%%%%%%%%%%%%%%%%%%%%%%%%%%%%%%%%%%%%%%%%%%%%%%%%%%%%%%
%\newpage{\pagestyle{empty}\cleardoublepage}
%\thispagestyle{empty}

%\begin{appendix}

%\chapter{\LARGE Título del Apéndice 1}
%\label{appendix:1}
%\input{apendice1.tex}

%\chapter{\LARGE Título del Apéndice 2}
%\label{appendix:2}
%\input{apendice2.tex}

%\end{appendix}

%%%%%%%%%%%%%%%%%%%%%%%%%%%%%%%%%%%%%%%%%%%%%%%%%%%%%%%%%%
\bibliographystyle{plain}
\bibliography{bibliografia}
%%%%%%%%%%%%%%%%%%%%%%%%%%%%%%%%%%%%%%%%%%%%%%%%%%%%%%%%%%

\end{document}
