\section{Algoritmo consturictivo voraz}

%%%%%%%%%%%%%%%%%%%%%%%%%%%%%%%%%%%%%%%%%%%%%%%%%%%%%%%%%%%%%%%%%%%%%%%%%%%%%%%%%%%%%%%%%%%%%%%%%%%
\section*{Un constructivo voraz}
Un algoritmo constructivo voraz muy sencillo para este problema parte del subconjunto, $S$, formado por las $m$ tareas con menores valores de $t_{0j}$ asignadas a los $m$ arrays que representan la secuenciaci\'on de tareas en las m\'aquinas. A continuaci\'on, a\~nade a este subconjunto, iterativamente, la tarea-m\'aquina-posici\'on que menor incremento produce en la funci\'on objetivo. El pseudoc\'odigo de este algoritmo se muestra a conti\-nuaci\'on.
%%%%%%%%%%%%%%%%%%%%%%%%%%%%%%%%%%%%%%
\begin{figure}[h!]
{\small
 \hrule \
 {\bf\small Algoritmo constructivo voraz}
 \hrule
\begin{center}
\begin{tabbing}
\ 1: Seleccionar la $m$ tareas $j_1, j_2,.., j_m$ con menores valores de $t_{0j}$ para ser introducidas en las \\ primeras posiciones de los arrays que forman la solucion $S$;\\
\ 2: S = \{A_1=\{j_1\}, A_2=\{j_2\},.., A_m=\{j_m\}\};\\
\ 3: {\bf rep}\={\bf eat}\\
\ 4: \> $S^* = S$;\\
\ 5: \> Obtener la tarea-maquina-posicion que minimiza el incremento del $TCT$;\\
\ 6: \> Insertarla en la posicion que corresponda y actualizar $S^*$;\\
\ 7: {\bf until} (todas las tareas han sido asignadas a alguna maquina)\\
\ 9: Devolver $S^*$;
\end{tabbing}
\end{center}
\hrule
}
\caption{Algoritmo constructivo voraz}
\label{constructivo}
\end{figure}

%%%%%%%%%%%%%%%%%%%%%%%%%%%%%%%%%%%%%%%%%%%%%%%%%%%%%%%%%%%%
%%%%%%%%%%%%%%%%%%%%%%%%%%%%%%%%%%%%%%%%%%%%%%%%%%%%%%%%%%%%
\section{Búsquedas Locales}


%%%%%%%%%%%%%%%%%%%%%%%%%%%%%%%%%%%%%%%%%%%%%%%%%%%%%%%%%%%%
%%%%%%%%%%%%%%%%%%%%%%%%%%%%%%%%%%%%%%%%%%%%%%%%%%%%%%%%%%%%
\section{GRASP}



%%%%%%%%%%%%%%%%%%%%%%%%%%%%%%%%%%%%%%%%%%%%%%%%%%%%%%%%%%%%
%%%%%%%%%%%%%%%%%%%%%%%%%%%%%%%%%%%%%%%%%%%%%%%%%%%%%%%%%%%%
\section{GVNS}

