    \section{Algoritmo consturictivo voraz}

%%%%%%%%%%%%%%%%%%%%%%%%%%%%%%%%%%%%%%%%%%%%%%%%%%%%%%%%%%%%%%%%%%%%%%%%%%%%%%%%%%%%%%%%%%%%%%%%%%%
\section*{Un constructivo voraz}
Un algoritmo constructivo voraz muy sencillo para este problema parte del subconjunto, $S$, formado por las $m$ tareas con menores valores de $t_{0j}$ asignadas a los $m$ arrays que representan la secuenciaci\'on de tareas en las m\'aquinas. A continuaci\'on, a\~nade a este subconjunto, iterativamente, la tarea-m\'aquina-posici\'on que menor incremento produce en la funci\'on objetivo. El pseudoc\'odigo de este algoritmo se muestra a conti\-nuaci\'on.
%%%%%%%%%%%%%%%%%%%%%%%%%%%%%%%%%%%%%%
\begin{figure}[h!]
{\small
 \hrule \
 {\bf\small Algoritmo constructivo voraz}
 \hrule
\begin{center}
\begin{tabbing}
\ 1: Seleccionar la $m$ tareas $j_1, j_2,.., j_m$ con menores valores de $t_{0j}$ para ser introducidas en las \\ primeras posiciones de los arrays que forman la solucion $S$;\\
\ 2: S = \{A_1=\{j_1\}, A_2=\{j_2\},.., A_m=\{j_m\}\};\\
\ 3: {\bf rep}\={\bf eat}\\
\ 4: \> $S^* = S$;\\
\ 5: \> Obtener la tarea-maquina-posicion que minimiza el incremento del $TCT$;\\
\ 6: \> Insertarla en la posicion que corresponda y actualizar $S^*$;\\
\ 7: {\bf until} (todas las tareas han sido asignadas a alguna maquina)\\
\ 9: Devolver $S^*$;
\end{tabbing}
\end{center}
\hrule
}
\caption{Algoritmo constructivo voraz}
\label{constructivo}
\end{figure}

%%%%%%%%%%%%%%%%%%%%%%%%%%%%%%%%%%%%%%%%%%%%%%%%%%%%%%%%%%%%
%%%%%%%%%%%%%%%%%%%%%%%%%%%%%%%%%%%%%%%%%%%%%%%%%%%%%%%%%%%%
\section{Búsquedas Locales}

Se han implementado 4 busquedas locales, las cuales se detallaran a continuación:
\begin{itemize}
  \item \textbf{Inserción Interna} - Trata de insertar cada tarea de una maquinas en todas las posiciones restantes de la misma
  \item \textbf{Inserción Externa} -Trata de insertar cada tarea de las maquinas en todas las posiciones posibles en otras maquinas
  \item \textbf{Intercambio Interno} - Trata de intercambiar todos las tareas de una maquian entre si
  \item \textbf{Intercambio Externo} - Trata de intercambiar todas las tareas de las maquinas por todas las tareas de otras
\end{itemize}
 
%%%%%%%%%%%%%%%%%%%%%%%%%%%%%%%%%%%%%%%%%%%%%%%%%%%%%%%%%%%%
%%%%%%%%%%%%%%%%%%%%%%%%%%%%%%%%%%%%%%%%%%%%%%%%%%%%%%%%%%%%
\section{GRASP}
Para el algoritmo GRASP empleado, tras experimentar con varios tanmaños de instancias (veces que corremos la fase constructiva) opte por 10 ya que era un buen balance entre buenos resultados y una ejecución rápida.

\begin{figure}[h!]
{\small
 \hrule \
 {\bf\small Algoritmo constructivo voraz}
 \hrule
\begin{center}
\begin{tabbing}
\textbf{Procedure} \= \textbf{GRASP} \\
\textbf{Begin} \\
\> Preprocesamiento \\
\> \textbf{Repeat} \\
\> \> Fase Constructiva(Solución); Usando para ello una lista de n candidatos \\
\> \> PostProcesamiento(Solución); Miramos optimos locales para obtener una nueva solución \\
\> \> Actualizar(Solución, MejorSolución); Si es mejor que la mejor encontrada, se guarda \\
\> \textbf{Until} (Se halla llegado a las iteraciones deseadas); \\
\textbf{End.} \\
\end{tabbing}
\end{center}
\hrule
}
\caption{Algoritmo constructivo grasp}
\label{constructivo}
\end{figure}


%%%%%%%%%%%%%%%%%%%%%%%%%%%%%%%%%%%%%%%%%%%%%%%%%%%%%%%%%%%%
%%%%%%%%%%%%%%%%%%%%%%%%%%%%%%%%%%%%%%%%%%%%%%%%%%%%%%%%%%%%
\section{GVNS}

En cuanto a las instancias (veces que corremos la fase constructiva) opte por 100, ya que era un buen número para obtener resultados y no tardaba una cantidad de tiempo excesivo (por no optimizar las busquedas locales tanto como podria, llevar  acabo las 1000 iteraciones simplemente llevaba demasiado tiempo).